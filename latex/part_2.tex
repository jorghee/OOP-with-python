\subsection{Método \texttt{updateContact}}
\subsubsection{Definición del Método}
\begin{verbatim}
\begin{lstlisting}[language=python]
def updateContact(self, newContact):
    for idx, saved in enumerate(self._contacts):
        if saved.nombre == newContact.nombre:
            self._contacts[idx] = newContact
            print("El contacto se actualizó exitosamente...\n")
            return
    print("No se encontró el contacto a actualizar...\n")
\end{lstlisting}
\end{verbatim}

\subsubsection{Descripción del Funcionamiento}
\begin{itemize}
    \item \textbf{Propósito:}
    Actualizar un contacto existente en la lista \texttt{\_contacts}.
    \item \textbf{Parámetros:}
    \texttt{newContact}: Un objeto de contacto que contiene la nueva información que se desea actualizar.
    \item \textbf{Flujo de Control:}
    \begin{itemize}
        \item \textbf{Iteración:} El método recorre la lista \texttt{\_contacts} usando \texttt{enumerate} para obtener tanto el índice como el objeto de contacto actual.
        \item \textbf{Comparación:} Dentro del bucle, compara el nombre (\texttt{nombre}) del contacto guardado (\texttt{saved}) con el nombre del nuevo contacto (\texttt{newContact}).
        \item \textbf{Actualización:} Si encuentra una coincidencia, reemplaza el contacto existente en la lista con el nuevo contacto (\texttt{newContact}).
        \item \textbf{Salida:} Imprime un mensaje de éxito y sale del método con \texttt{return} para evitar iteraciones innecesarias.
        \item \textbf{No Encontrado:} Si no se encuentra una coincidencia tras iterar toda la lista, imprime un mensaje indicando que no se encontró el contacto.
    \end{itemize}
\end{itemize}

\subsection{Método \texttt{deleteContact}}
\subsubsection{Definición del Método}
\begin{verbatim}
\begin{lstlisting}[language=python]
def deleteContact(self, pattern):
    for contact in self._contacts:
        name = contact.nombre
        isName = self._match(name.lower(), pattern.lower())
        if isName:
            self._contacts.remove(contact)
            print("El contacto se eliminó exitosamente...\n")
            return
    print("No se encontró el contacto a eliminar...\n")
\end{lstlisting}
\end{verbatim}

\subsubsection{Descripción del Funcionamiento}
\begin{itemize}
    \item \textbf{Propósito:}
    Eliminar un contacto de la lista \texttt{\_contacts} que coincida con un patrón dado.
    \item \textbf{Parámetros:}
    \texttt{pattern}: Un patrón de texto que se usará para buscar coincidencias en los nombres de los contactos.
    \item \textbf{Flujo de Control:}
    \begin{itemize}
        \item \textbf{Iteración:} El método recorre la lista \texttt{\_contacts}.
        \item \textbf{Comparación:} Convierte los nombres de los contactos y el patrón a minúsculas para una comparación insensible a mayúsculas/minúsculas.
        \item \textbf{Coincidencia:} Utiliza un método auxiliar \texttt{\_match} para verificar si el nombre coincide con el patrón.
        \item \textbf{Eliminación:} Si encuentra una coincidencia, elimina el contacto de la lista \texttt{\_contacts}.
        \item \textbf{Salida:} Imprime un mensaje de éxito y sale del método con \texttt{return}.
        \item \textbf{No Encontrado:} Si no se encuentra una coincidencia tras iterar toda la lista, imprime un mensaje indicando que no se encontró el contacto.
    \end{itemize}
\end{itemize}