\subsection{Método \texttt{updateContact}}
\subsubsection{Definición del Método}

\begin{minted}[bgcolor=background]{python}
def updateContact(self, newContact):
    for idx, saved in enumerate(self._contacts):
        if saved.nombre == newContact.nombre:
            self._contacts[idx] = newContact
            print("El contacto se actualizó exitosamente...\n")
            return
    print("No se encontró el contacto a actualizar...\n")
\end{minted}

\subsubsection{Descripción del Funcionamiento}
\begin{itemize}
    \item \textbf{Propósito:}
    Actualizar un contacto existente en la lista \texttt{\_contacts}.
    \item \textbf{Parámetros:}
    \texttt{newContact}: Un objeto de contacto que contiene la nueva información que se desea actualizar.
    \item \textbf{Flujo de Control:}
    \begin{itemize}
        \item \textbf{Iteración:} El método recorre la lista \texttt{\_contacts} usando \texttt{enumerate} para obtener tanto el índice como el objeto de contacto actual.
        \item \textbf{Comparación:} Dentro del bucle, compara el nombre (\texttt{nombre}) del contacto guardado (\texttt{saved}) con el nombre del nuevo contacto (\texttt{newContact}).
        \item \textbf{Actualización:} Si encuentra una coincidencia, reemplaza el contacto existente en la lista con el nuevo contacto (\texttt{newContact}).
        \item \textbf{Salida:} Imprime un mensaje de éxito y sale del método con \texttt{return} para evitar iteraciones innecesarias.
        \item \textbf{No Encontrado:} Si no se encuentra una coincidencia tras iterar toda la lista, imprime un mensaje indicando que no se encontró el contacto.
    \end{itemize}
\end{itemize}

\subsection{Método \texttt{deleteContact}}
\subsubsection{Definición del Método}

\begin{minted}[bgcolor=background]{python}
def deleteContact(self, pattern):
    for contact in self._contacts:
        name = contact.nombre
        isName = self._match(name.lower(), pattern.lower())
        if isName:
            self._contacts.remove(contact)
            print("El contacto se eliminó exitosamente...\n")
            return
    print("No se encontró el contacto a eliminar...\n")
\end{minted}

\subsubsection{Descripción del Funcionamiento}
\begin{itemize}
    \item \textbf{Propósito:}
    Eliminar un contacto de la lista \texttt{\_contacts} que coincida con un patrón dado.
    \item \textbf{Parámetros:}
    \texttt{pattern}: Un patrón de texto que se usará para buscar coincidencias en los nombres de los contactos.
    \item \textbf{Flujo de Control:}
    \begin{itemize}
        \item \textbf{Iteración:} El método recorre la lista \texttt{\_contacts}.
        \item \textbf{Comparación:} Convierte los nombres de los contactos y el patrón a minúsculas para una comparación insensible a mayúsculas/minúsculas.
        \item \textbf{Coincidencia:} Utiliza un método auxiliar \texttt{\_match} para verificar si el nombre coincide con el patrón.
        \item \textbf{Eliminación:} Si encuentra una coincidencia, elimina el contacto de la lista \texttt{\_contacts}.
        \item \textbf{Salida:} Imprime un mensaje de éxito y sale del método con \texttt{return}.
        \item \textbf{No Encontrado:} Si no se encuentra una coincidencia tras iterar toda la lista, imprime un mensaje indicando que no se encontró el contacto.
    \end{itemize}
\end{itemize}

\subsection{\textcolor{red}{Probando la funcionalidad del programa Agenda telefónica}}
La clase Principal es la que implementan la interacción con el usuario, se ha tratado de simular una shell interactiva, como si fuera escribir comandos dentro de un emulador de terminal.

\begin{minted}[bgcolor=background]{python}
def menu(self):

  print("Welcome to address book!")
  print("""
  ---------------------------------------------------------------------
  |   1. search -> search by name                                     |
  |   2. insert -> insert the new contact                             |
  |   3. update -> update the contact                                 |
  |   4. delete -> delete the contact                                 |
  |   5. list -> list the agenda's contacts                           |
  |   6. save -> save the contacts                                    |
  |   7. exit -> exit the program                                     |
  ---------------------------------------------------------------------
  """)

  while True:
    option = input("agenda> ")
    option = option.split()

    if option[0] == "search":
      if len(option) == 2:
        contact = self._agenda.getContact(option[1])
        print(contact)
      else:
        print("Ingrese: search <nombre>\n")
    elif option[0] == "insert":
      # Recibe 4 argumentos, de lo contrario ocurre un comportamiento no esperado
      if len(option) == 5:
        newContact = Contacto(option[1], option[2], option[3], option[4])
        self._agenda.insertContact(newContact)
      else:
        print("Ingrese: insert <nombre> <numero> <direccion> <relacion>\n")
    elif option[0] == "update":
      if len(option) == 5:
        newContact = Contacto(option[1], option[2], option[3], option[4])
        self._agenda.updateContact(newContact)
      else:
        print("Ingrese: update <nombre> <numero> <direccion> <relacion>\n")
    elif option[0] == "delete":
      if len(option) == 2:
        self._agenda.deleteContact(option[1])
      else:
        print("Ingrese: delete <nombre>\n")
    elif option[0] == "list":
      if len(option) == 1:
        self._agenda.listAgenda()
      else:
        print("Ingrese: list\n")
    elif option[0] == "save":
      if len(option) == 1:
        self._agenda.guardar_agenda()
      else:
        print("Ingrese: save\n")
    elif option[0] == "exit":
      break
    else:
      print("Comando no encontrado\n")
\end{minted}

Para poder visualizar la funcionalidad en linea de comandos, se ha incluido un video en el siguiente link donde se observa las inserciones que se solicitan.
\singlespacing
\textcolor{red}{\textbf{Link del video YouTube:}} \href{https://youtu.be/TQ0Thgi5O5g}{Agenda telefónica}
